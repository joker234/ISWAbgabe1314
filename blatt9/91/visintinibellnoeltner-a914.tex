%{{{ Heading Stuff

%{{{ Documentclass, usepackages
%\documentclass[ngerman, landscape]{scrartcl}
\documentclass[ngerman]{scrartcl}
\usepackage[ngerman]{babel}
%}}}

%{{{ Titel, Author, etc...

\newcommand{\blattnummer}{9}												%% Nummer des Übungsblattes angeben
\newcommand{\aufgabe}{1}
\newcommand{\myauthor}{Moritz Nöltner, Johannes Visintini, Philip Bell}
\newcommand{\mytutor}{Amin Kiem}
\newcommand{\mytitle}{Abgabe für ISW Blatt \blattnummer}
\title{\mytitle}
\author{\myauthor}
%}}}

%{{{ Kopf und Fußzeilen einrichten
\usepackage{fontspec, fancyhdr}
\pagestyle{fancy}

\lhead{\bfseries \mytitle}
%%\chead{\mytitle}
\rhead{\bfseries \myauthor}

\lfoot{\bfseries \today}
\cfoot{Tutor: \mytutor}
\rfoot{\bfseries \thepage}

\renewcommand{\headrulewidth}{1pt}
\renewcommand{\footrulewidth}{1pt}
%\textheight = 592pt	%% Standardwert
\textheight = 650pt 	%% Für Hochformat
%%\textheight = 450pt 	%% Für Querformat
%}}}

%{{{ Longtable

\usepackage{multirow}
\usepackage{array}
\usepackage{longtable}
\usepackage{pdfpages}
%}}}

%{{{ Minted (Highlighter für Code)

%% \usepackage{minted}
%% \definecolor{bg}{rgb}{0.95,0.95,0.95}
%% \newminted[Java]{java}{numbersep=5pt, bgcolor=bg, gobble=0, linenos, tabsize=4}
%}}}


%}}}


\begin{document}
	\setcounter{section}{\blattnummer}
	\setcounter{subsection}{\aufgabe-1}
	\subsection{Klassendiagramm EMF Movie Manager}
	\setcounter{subsubsection}{3}
	\subsubsection{Erklärung}

	\begin{description}
	\item [sortiert] TreeViewerColumnSorter sortiert die Anzeige im Haupfenster. Dafür wird für jedes Anzeigeelement ein TreeViewerColumnSorter in MoviesView gespeichert.
	\item [liefert Label] Der CustomColumnLabelProvider liefert die Label für die Filme. Dafür wird ein CustomColumnLabelProvider in MoviesView gespeichert.
	\item [liefert Inhalt] MoviesView erstellt einen Tree, der MoviesContentProvider wird dem Tree als ContentProvider zugewiesen.
	\item [wird verwaltet von] MovieViewHandler erzeugt die Umgebung und öffnet dann ein MoviesView.
	\end{description}

	\setcounter{subsubsection}{5}
	\subsubsection{Gegenüberstellung}
	Der EMF-MovieManager verwendet zur Anzeige Graphische Elemente, während der Swing-MovieManager nur eine Tabelle (JTable benutzt).\\
	Analog zum TreeViewerColumnSorter der EMF-Version benötig die Jtable den TableRowSorter, die Entsprechung des MoviesContentProviders ist das MovieTableModel, dass die Anzeige mit Daten versorgt.\\
	Während beim EMF-Modell die Anzeigeelemente in einem Tree verwaltet werden und graphischer Natur sind, nutzt die Swing-Variante nur die Textfelder der Tabelle und sonst Standardelemente wie JButton und JTextfield.


\end{document}
