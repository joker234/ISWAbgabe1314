\documentclass{scrartcl}
\usepackage[left=1cm, right=1cm, top=1cm, bottom=1cm]{geometry}
\usepackage[ngerman]{babel}
\usepackage{fontspec}


\usepackage{multirow}
\usepackage{array}
\usepackage{pdfpages}


\begin{document}

\author{Johannes Visintini, Philip Bell, Moritz Nöltner}
\title{Einführung in Software Engineering}
\subtitle{WS 13/14, Aufgabe 9.5}
\maketitle

\textheight = 650pt

\subsection*{Was war bei der Projektdurchführung besonders schwierig für Sie
	bzw. hat nicht so gut funktioniert? Was hat gut funktioniert?}
\begin{itemize}
	\item Wir waren nur drei Teilnehmer, da die Kommunikation mit dem
		vierten Team-Teilnehmer nicht funktioniert hat. Dadurch hatten wir
		einen höheren Arbeitsaufwand, es war aber einfacher sich
		abzusprechen.
	\item Teils chaotische Organisation in der Gruppe. Die Arbeit wurde aber
		trotzdem immer erledigt.
	\item Die Aufgabenstellungen waren oft unverständlich oder nicht genau
		spezifiziert. Dadurch war es schwierig die Projekte gemäß den
		Anforderungen durchzuführen.
\end{itemize}

\subsection*{Was haben Sie bisher gelernt? Geben Sie 3 gelernte
	Techniken/Praktien an, die Sie als wichtig für die systematische Entwicklung
	ansehen. Begründen Sie, warum das für Sie wichtig ist.}
\begin{itemize}
	\item \textbf{Diagramme anzulegen} (Klassen-, Use-Case-, Sequenzdiagramme,
		etc.)\\
		Die Diagramme verschaffen dem Team eine Struktur an der man sich
		orientieren sollte, wenn man zusammen Software entwickeln will. Z.B.
		erkennt man am Sequenzdiagramm ob sich Abläufe nicht einfacher
		implementieren lassen.
	\item \textbf{Team-Präsentationen erstellen}\\
		Team-Präsentationen sind wichtig um Ausschreibungen für Projekte für
		sich als Entwickler-Team zu entscheiden.
	\item \textbf{Verwendung einer Entwicklungsumgebung und eines
		Versionsverwaltungssystem}\\
		Um Parallel an Software zu arbeiten sollte man eine
		Entwicklungsumgebung (die für einheitliche Programmierrichtlinien (z.B.
		CheckStyle) sorgen kann) und eine Versionsverwaltungssystem (das
		paralleles arbeiten ermöglicht) einsetzen.
\end{itemize}

\subsection*{Reicht das bisher Gelernte? Beschreiben Sie, welches Wissen Ihnen
	noch fehlt um in einem Team Software zu entwickeln.}
\begin{itemize}
	\item[] Nein. Um Einzuschätzen ob die Planung hilfreich war, sollte man das
		Projekt nun umsetzen um zu sehen ob die Planung erfolgreich war. Es
		fehlt die Übung mit Planung \textbf{und} Durchsetzung. Ein Projekt
		bringt einen nicht weiter. Es fehlen mehr bzw. schwerere Projekte.
\end{itemize}

\end{document}
