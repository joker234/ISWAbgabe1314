%{{{ Heading Stuff
\documentclass[ngerman]{scrartcl}
\usepackage[ngerman]{babel}
\usepackage{fontspec}

\usepackage{minted}
\definecolor{bg}{rgb}{0.95,0.95,0.95}
\newminted[C++]{cpp}{numbersep=5pt, bgcolor=bg, gobble=0, linenos, tabsize=4}
\newminted[Java]{java}{numbersep=5pt, bgcolor=bg, gobble=0, linenos, tabsize=4}
%}}}


%{{{
\begin{document}

\author{Moritz Nöltner}
\title{Einführung in Software Engineering}
\subtitle{WS 13/14, Aufgabe 1.3}
\maketitle


\section{Dynamisch entdeckbare Softwarefehler}
\subsection{Out of bounds Zugriff}
Greift man auf ein Array zu, und überprüft dabei nicht, ob der Index innerhalb der Arraygrenzen liegt, so kann es sein, dass dieser es nicht ist. In diesm Fall kann je nach Architektur entweder ein Segmentierungsfehler ausgelöst werden, der umliegende Speicher korrumpiert werden, oder wie bei Java eine Out-Out-Bounds-Exception ausgelöst werden.
\begin{Java}
package tests;
import java.util.Scanner;

public class Main {
	public int[] foo;
	public Main(){
		this.foo=new int[10];
	}

	public static void main(String[] args) {
		Main bar=new Main();
		System.out.println("Bitte geben sie an, welcher Eintrag auf 10"
			+ "gesetzt werden soll:");
		Scanner sc = new Scanner(System.in);
		int num = sc.nextInt();
		bar.foo[num]=10;	// Fehler, falls num > 9
		for(int i=0; i<10; i++)
		{
			System.out.println(bar.foo[i]);
		}
	}
}
\end{Java}


\subsection{Nullpointerdereferenzierung}
Wenn man beispielsweise die get()-Funktion einer Collection verwendet, kann es vorkommen, dass man als Ergebnis "null"  geliefert bekommt. Vergisst man dann, diesen Fall abzufangen, löst man je nach System einen Segmentierungsfehler aus, greift auf fremden Speicher zu, oder löst eine Exception aus.

\section{Statisch entdeckbare Softwarefehler}
\subsection{Zuweisung statt Vergleich}
Oftmals kommt es vor, dass man statt einem logischen Vergleich eine Zuweisung vornimmt.\\
Beispielsweise:\\
\begin{Java}
	...
	if(foo = bar) { // statt if(foo == bar) {
		// do something;
	}
	...
\end{Java}

\subsection{Fehlendes volatile}
Wird eine Variable in einem Interrupt verändert, und der Zugriff auf diese Variable nicht mit volatile Synchronisiert, so kann es dazu kommen, dass veraltete Daten weiter genutzt werden. Werden diese dann zurückgeschrieben, kann es sein, dass neuere Daten sogar überschrieben werden.

\section{Zusammenfassung}

\begin{tabular}{|p{3cm}|p{5cm}|p{6cm}|}
	\hline
	\textbf{Softwarefehler} & \textbf{Erklärung} & \textbf{Statisch/Dynamisch + Begründung} \\
	\hline
	Toter Code & Code, der nie ausgeführt wird & Statisch, kann Compiler entdecken \\
	\hline
	Access out of Bounds & Zugriff auf Array ausserhalb der Arraygrenzen & Dynamisch, wenn man den Index durch Benutzereingabe bekommt \\
	\hline
	Nullpointerdereferenzierung & Zugriff auf Methoden eines nicht initialisierten Objekts & Dynamisch, weil oft erst zur Laufzeit bekannt ist, ob Objekt null ist. \\
	\hline
	Vergleich statt Zuweisung & Statt "'=="' wird nur "'="' verwendet & Statisch, eigentlich immer sollte in if(...) eine logische Operation stehen. \\
	\hline
	Verwendung von Java & Es wird Java statt einer "'richtigen"' Programmiersprache verwendet & Statisch. Ein guter C++-Compiler erkennt das. \\
	\hline
	Fehlendes "'volatile"' & Veränderte Daten werden weitergenutzt & Statisch. Man könnte prüfen, ob auf Daten in einer ISR zugegriffen wird. \\
	\hline
\end{tabular}

\end{document}
%}}}







