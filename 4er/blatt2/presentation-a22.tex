\documentclass{beamer} %% normal document
%\documentclass[notes]{beamer} %% notes in normal document
%\documentclass[draft,notes]{beamer} %% draft with notes
%\documentclass[handout]{beamer} %% handout

\usepackage{fontspec, unicode-math, caption}
\usepackage[ngerman]{babel}

%% usefull for handout with blank lines
%\usepackage{handoutWithNotes}
%\pgfpagesuselayout{2 on 1 with notes}[a4paper,border shrink=5mm]

%% usefull for presentation
%\setbeameroption{show notes on secondscreen=left}

\definecolor{bettergreen}{rgb}{.1,.7,.1}

\usetheme{Dresden}
\usecolortheme[named=bettergreen]{structure}
\useoutertheme{split}
\setbeamertemplate{caption}[numbered]
\captionsetup{labelformat=simple,font=scriptsize,labelfont=scriptsize}

% puts Frame numbers in Dresden template
\newcommand*\oldmacro{}%
\let\oldmacro\insertshorttitle%
\renewcommand*\insertshorttitle{%
  \oldmacro\hfill%
  \insertframenumber\,/\,\inserttotalframenumber}

\title[]{}
\author{
	Johannes Visintini, Philip Bell,\\
	Moritz Nöltner, Andrii Soliar
}
\institute[IFI]{
	Vorlesung: Einführung in Software Engineering\\
	Institut für Informatik\\
	Universität Heidelberg
	}


\begin{document}

\begin{frame}[plain]
\titlepage
	\note{ }
\end{frame}

\begin{frame}{Inhalt}
\vbox{
\tableofcontents}
	\note{ }
\end{frame}

\section{Teampräsentation}
\begin{frame}{Teampräsentation}
	\center{\huge Die ASDF-Group}
	\vspace{2em}
	\begin{itemize}
		\item Philip Bell\\
			B.Sc. Mathematik, 5. Fachsemester
		\item Moritz Nöltner\\
			B.Sc. Angewandte Informatik, 5. Fachsemester
		\item Andrii Soliar\\
			B.Sc. Angewandte Informatik, 4. Fachsemester
		\item Johannes Visintini\\
			B.Sc. Angewandte Informatik, 5. Fachsemester
	\end{itemize}
\end{frame}

\section{Funktionalitäten}
\begin{frame}{Funktionalitäten}
	\begin{itemize}\pause
		\item \textbf{Projekt} (Datensammlung)\\
			Anlegen
			\note{Daten werden automatisch in ~/.moviemanager gespeichert.}\pause
		\item \textbf{Movie} (Film)\\
			Anlegen, in Collection einfügen, Performer hinzufügen\\\pause
			Online Similar (2 Filme verknüpfen)\pause
		\item \textbf{Performer} (Schauspieler)\\
			Anlegen, zu Movie hinzufügen\pause
		\item \textbf{Collection} (Filmsammlung)\\
			Anlegen, Movie einfügen\pause
		\item \textbf{Kurzübersicht}\\
			über Filme und Schauspieler
	\end{itemize}
\end{frame}

\section{zusätzliche Funktionalität}
\begin{frame}{zusätzliche Funktionalität}
	\begin{itemize}
		\item bisher wird gespeichert ob der Film verliehen wird\pause
		\item es soll eine Klasse „Customer“ eingeführt werden
			\begin{itemize}
				\item firstname
				\item lastname
				\item adress
				\item mail adress
				\item gender
			\end{itemize}\pause
		\item es wird nun nicht nur gespeichert \textbf{ob} der Film verliehen
			ist, sondern \textbf{wer} ihn ausgeliehen hat\pause
		\item Fancy fancy fancy
	\end{itemize}
\end{frame}

\section{}
\begin{frame}
	\center{\Huge ENDE}
\end{frame}

\end{document}
