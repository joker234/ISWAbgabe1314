%{{{ Heading Stuff

%{{{ Documentclass, usepackages
%\documentclass[ngerman, landscape]{scrartcl}
\documentclass[ngerman]{scrartcl}
\usepackage[ngerman]{babel}
%}}}

%{{{ Titel, Author, etc...

\newcommand{\blattnummer}{7}												%% Nummer des Übungsblattes angeben
\newcommand{\myauthor}{Moritz Nöltner, Johannes Visintini, Philip Bell}
\newcommand{\mytutor}{Amin Kiem}
\newcommand{\mytitle}{Abgabe für ISW Blatt \blattnummer}
\title{\mytitle}
\author{\myauthor}
%}}}

%{{{ Kopf und Fußzeilen einrichten
\usepackage{fontspec, fancyhdr}
\pagestyle{fancy}

\lhead{\bfseries \mytitle}
%%\chead{\mytitle}
\rhead{\bfseries \myauthor}

\lfoot{\bfseries \today}
\cfoot{Tutor: \mytutor}
\rfoot{\bfseries \thepage}

\renewcommand{\headrulewidth}{1pt}
\renewcommand{\footrulewidth}{1pt}
%\textheight = 592pt	%% Standardwert
\textheight = 650pt 	%% Für Hochformat
%%\textheight = 450pt 	%% Für Querformat
%}}}

%{{{ Longtable

\usepackage{multirow}
\usepackage{array}
\usepackage{longtable}
\usepackage{pdfpages}
%}}}

%{{{ Minted (Highlighter für Code)

%% \usepackage{minted}
%% \definecolor{bg}{rgb}{0.95,0.95,0.95}
%% \newminted[Java]{java}{numbersep=5pt, bgcolor=bg, gobble=0, linenos, tabsize=4}
%}}}


%}}}


\begin{document}
	\setcounter{section}{\blattnummer}
	\subsection{Klassendiagramm}
	\setcounter{subsubsection}{2}
	\subsubsection{}

	\begin{description}
		\item [called\_by\_to\_show\_ui:] Die Klasse Main instanziiert ein MovieUI-Objekt, um die GUI anzuzeigen
		\item [painted\_on\_by:] MovieUI zeichnet sich auf einen JFrame
		\item [used\_by:] JButton und JTextField werden von MovieUI angezeigt um die Möglichkeit, einen Film zuzufügen, und die Möglichkeit, die Anzeige zu filtern, zugänglich zu machen.
		\item [sorts:] TableRowSorter wird verwendet, um die Tabelle zu ordnen.
		\item [operates:] Die Interna der JTable werden durch MovieTableModel verwaltet, dafür speichert die JTable ihr MovieTableModel ab.
		\item [realises:] MovieTableModel erbt von AbstractTableModel, und erzeugt ein nicht abstraktes, auf MovieManager zugeschneidertes TableModel.
		\item [provides\_data:] Das MovieTableModel bekommt seine Daten vom MovieManager. Es speichert dazu eine Instanz des MovieManagers ab.
		\item [managed\_by:] Die Movies werden im MovieManager gespeichert und verwaltet.
		\item [owned\_by:] MovieUI speichert auch das MovieTableModel. Der Zweck ist nicht bekannt, denn eigentlich bietet JTable auch die Funktion "`getTableModel()"' an, mit der sich auch auf MovieTableModel zugreifen ließe.
	\end{description}



\end{document}
