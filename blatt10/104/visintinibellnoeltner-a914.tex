%{{{ Heading Stuff

%{{{ Documentclass, usepackages
%\documentclass[ngerman, landscape]{scrartcl}
\documentclass[ngerman]{scrartcl}
\usepackage[ngerman]{babel}
%}}}

%{{{ Titel, Author, etc...

\newcommand{\blattnummer}{10}												%% Nummer des Übungsblattes angeben
\newcommand{\aufgabe}{1}
\newcommand{\myauthor}{Moritz Nöltner, Johannes Visintini, Philip Bell}
\newcommand{\mytutor}{Amin Kiem}
\newcommand{\mytitle}{Abgabe für ISW Blatt \blattnummer}
\title{\mytitle}
\author{\myauthor}
%}}}

%{{{ Kopf und Fußzeilen einrichten
\usepackage{fontspec, fancyhdr}
\pagestyle{fancy}

\lhead{\bfseries \mytitle}
%%\chead{\mytitle}
\rhead{\bfseries \myauthor}

\lfoot{\bfseries \today}
\cfoot{Tutor: \mytutor}
\rfoot{\bfseries \thepage}

\renewcommand{\headrulewidth}{1pt}
\renewcommand{\footrulewidth}{1pt}
%\textheight = 592pt	%% Standardwert
\textheight = 650pt 	%% Für Hochformat
%%\textheight = 450pt 	%% Für Querformat
%}}}

%{{{ Longtable

\usepackage{multirow}
\usepackage{array}
\usepackage{longtable}
\usepackage{pdfpages}
%}}}

%{{{ Minted (Highlighter für Code)

%% \usepackage{minted}
%% \definecolor{bg}{rgb}{0.95,0.95,0.95}
%% \newminted[Java]{java}{numbersep=5pt, bgcolor=bg, gobble=0, linenos, tabsize=4}
%}}}


%}}}

\newcommand{\f}{Functional Requirements}
\newcommand{\n}{Nonfunctional Requirements}
\newcommand{\g}{Glossary}

\begin{document}
	\setcounter{section}{\blattnummer}
	\setcounter{subsection}{\aufgabe-1}
	\subsection{Klassendiagramm EMF Movie Manager}
	\setcounter{subsubsection}{4}
	\subsubsection{Einführung Sprint}

	%{{{ Testfalltabelle

	\begin{tiny}

	\begin{longtable}{|l|l|}
		\hline
		Artefakt & Zuordnung im ReqSpecDocument\\
		\hline
		\hline
		Erhobene Defizite & \f\\
		\hline
		Geschäftsprozessmodelle & \f\\
		\hline
		Protokolle & \f\\
		\hline
		Rollenbeschreibungen & \f\\
		\hline
		Aufgabenbeschreibungen & \f\\
		\hline
		Glossareinträge & \g\\
		\hline
		Qualitätskriterien von Aufgaben & \n\\
		\hline
		Softwaredokumentation & \n\\
		\hline
		Lesetechnik & \n\\
		\hline
		Problem/Fehler/Korrekturliste & \n\\
		\hline
		Korrigierte SW-Dokumentation & \n\\
		\hline
		NFRs & \n\\
		\hline
		Domänendatendiagramm & \f\\
		\hline
		Datenbeschreibung & \f\\
		\hline
		Systemverantwortlichkeitenübersicht & \f\\
		\hline
		Aktivitätsdiagramme für IST und SOLL & \f\\
		\hline
		Ist- und Soll-Beschreibung & \f\\
		\hline
		Qualitätskriterien für die Domäne & \n\\
		\hline
		Qualitätskriterien für die Architektur & \n\\
		\hline
		Globale Qualitätskriterien für funktionale Anforderungen & \n\\
		\hline
		Systemverantwortlichkeiten & \n\\
		\hline
		QS-Plan (z.B.: wer, was, wie intensiv, Priorisierung) & \n\\
		\hline
		Systemtestplan & \n\\
		\hline
		Usecasebeschreibung & \f\\
		\hline
		Systemverantwortlichkeitenbeschreibung & \f\\
		\hline
		Domänendaten & \f\\
		\hline
		Systemfunktionsbeschreibung & \f\\
		\hline
		Nutzungsdiagramm & \f\\
		\hline
		Szenarien & \f\\
		\hline
		Qualitätskriterien für Usecases & \n\\
		\hline
		Qualitätskriterien für Systemfunktionen & \n\\
		\hline
		UI-Struktur-Diagramm & \f\\
		\hline
		Interaktionsdatendiagramm (verfeinertes Datendiagramm)& \f\\
		\hline
		Funktionale und nichtfunktionale Anforderungen & \n \& \f\\
		\hline
		Systemtestspezifikation & \n\\
		\hline
		Erstimplementierung & \f\\
		\hline
		Usabilitytestplan & \n\\
		\hline
		Usabilitytestspezifikation & \n\\
		\hline
		Prototyps & \f\\
		\hline
		ERD, UC, Systemfunktionsbeschreibungen & \f\\
		\hline
		Analyseklassendiagramm & \f\\
		\hline
		Sequenzdiagramme & \f\\
		\hline
		Dialog Text [sic.] & \f\\
		\hline
		Dialog Zustandsdiagramm [sic.] & \f\\
		\hline
		Hilfsfunktionen & \f\\
		\hline
		Philip Bell & \g\\
		\hline
		Johannes Visintini & \n\\
		\hline
	\end{longtable}
	\end{tiny}
	%}}}


\end{document}
