\documentclass{scrartcl}

\usepackage[ngerman]{babel}
\usepackage[T1]{fontenc}
\usepackage[utf8]{inputenc}
\usepackage{textcomp}
%\usepackage{ulem}
%\usepackage{a4wide}
\usepackage[top=2cm, bottom=2cm, left=1cm, right=1cm]{geometry}


\begin{document}
\title{ISW - Aufgabe 2.4b}
\author{Johannes Visintini}
\maketitle

\section*{Aufgabe 2.4b}

\begin{tabular}{|p{3.5cm}|p{2.5cm}|p{3.5cm}|p{2.5cm}|p{2cm}|p{2.5cm}|}\hline
	Anzahl … \hspace{4em} einer Klasse &
		Quelltextzeilen &
		ø Quelltextzeilen pro Operation &
		ø Parameter &
		Attribute &
		Operationen\\\hline
	im Analysetool &
		\tiny{Lines of Code} &
		\tiny{Average Lines Of Code Per Method} &
		\tiny{Average Number of Parameters} &
		\tiny{Number of Fields} &
		\tiny{Number of Methods}\\\hline
	Gender &
		4 &
		0 &
		0 &
		2 &
		0\\\hline
	Performer &
		61 &
		4.0 &
		0.45 &
		6 &
		11\\\hline
	Movie &
		76 &
		4.64 &
		0.46 &
		5 &
		13\\\hline
	MovieManager &
		30 &
		27 &
		1 &
		0 &
		1\\\hline
\end{tabular}\\
{\tiny(statt „Lines of Code“ könnte auch „Average Lines Of Code Per Method
$\cdot$ Number of Methods“ gemeint gewesen sein)}\\

Das \textit{Gender} sich stark von den anderen Klassen unterscheidet ist
klar, da dies nur ein enum ist und daher nur die beiden Möglichkeiten
beinhalten muss. \textit{Performer} und \textit{Movie} hingegen sind sich
eigentlich in allen Punkten relativ ähnlich, da sie beide eine richtige
Klasse darstellen. \textit{MovieManager} ist nur eine Util-Klasse, da
diese nur die auszuführende main-Funktion enthält und hat deshalb auch nur
eine Methode. Daher sind die Durchschnittswerte auch nicht
Aussagekräftig.\\

Durch das Erheben von Metriken muss sich der Programmierer mehr Gedanken
um seinen Code machen. Dies hat den Nachteil dass der Programmierer so
programmiert dass die Metriken „schön“ aus sehen. Es kann aber auch den
Vorteil haben das ungewöhnliche Zahlen in der Metrik auf Fehler hindeuten
können.\\
Das gleiche gilt für Codeprüfer (Revisoren).\\
Ein weiterer Nach-/Vorteil ist das ein Chef überprüfen kann wieviel Zeilen Attribute, Code der Programmierer pro Tag schafft ;)

\end{document}
