%{{{ Heading Stuff
\documentclass[ngerman, landscape]{scrartcl}
\usepackage[landscape, left=0cm, right=0cm, top=1cm, bottom=1cm]{geometry}
\usepackage[ngerman]{babel}
\usepackage{fontspec}


\usepackage{multirow}
\usepackage{array}
\usepackage{longtable}
\usepackage{pdfpages}

%%\documentclass[ngerman]{scrartcl}
%%\usepackage[ngerman]{babel}
%%\usepackage{fontspec}

\usepackage{minted}
\definecolor{bg}{rgb}{0.95,0.95,0.95}
\newminted[C++]{cpp}{numbersep=5pt, bgcolor=bg, gobble=0, linenos, tabsize=4}
\newminted[Java]{java}{numbersep=5pt, bgcolor=bg, gobble=0, linenos, tabsize=4}
%}}}


%{{{
\begin{document}

\author{Moritz Nöltner}
\title{Einführung in Software Engineering}
\subtitle{WS 13/14, Aufgabe 2.1}
\maketitle


\textheight = 650pt
\begin{longtable}{|p{8cm}|p{8cm}|p{8cm}|}
%%\begin{tabular}{|p{3cm}|p{5cm}|p{6cm}|}
	\hline
	\textbf{Bezeichnung} & \textbf{Funktion in Anwendung} & \textbf{Beschreibung} \\
	\hline
	\hline
	Neues Projekt anlegen & In der Navigatoransicht das Kontextmenu öffnen, dann die Option Other >>  New Project wählen. & Als Gliederungselement zur Organisation von Filmdaten können in der Navigatoransicht neue Projekte angelegt werden \\
	\hline
	Neues Objekt Movie anlegen & In der Navigatoransicht das Kontextmenu öffnen, die Option New Model Element wählen \& auf Movie doppelklicken. & Als Gliederungselement zur Organisation von Filmdaten könnenn in der Navigatoransicht neue Filme angelegt werden \\
	\hline
	Neues Objekt MovieCollection anlegen & In der Navigatoransicht das Kontextmenu öffnen, die Option New Model Element wählen \& auf MovieCollection doppelklicken. & Als Gliederungselement zur Organisation von Filmdaten könnenn in der Navigatoransicht neue Filme angelegt werden \\
	\hline
	Neues Objekt Performer anlegen & In der Navigatoransicht das Kontextmenu öffnen, die Option New Model Element wählen \& auf Performer doppelklicken. & Als Gliederungselement zur Organisation von Filmdaten könnenn in der Navigatoransicht neue Filme angelegt werden \\
	\hline
	Movie bewerten & In der Detailansicht eines Movies auf auf dem Drop-Down-Menue "Rating" die gewünschte Bewertung anklicken & Jedem Film  kann eine Bewertung zugeordnet werden. \\
	\hline
	Performer zu Movie zufügen & In der Detailansicht eines Movies unter Performer auf das Icon mit einem Menchenkopf und dem Kettensymbol klicken, dann in dem sich öffnenden Fenster den entsprechenden Performer auswählen. & So kann man angeben, welche Performer in einem Movie mitgespielt haben \\
	\hline
	Movie in Collection einfügen & In der Detailansicht eines Movies unter Collection auf das Icon mit einer Kiste und dem Kettensymbol klicken, dann in dem sich öffnenden Fenster die entsprechende Collection auswählen. & So kann man angeben, in welchen Collections ein Movie enthalten ist \\
	\hline
	Online Similar zu Movie zufügen & In der Detailansicht eines Movies unter Online Similar auf das Icon mit einem Filmstreifen und dem Kettensymbol klicken, dann in dem sich öffnenden Fenster das entsprechende Movie auswählen. & So kann man angeben, in welchen Online Similar es zu einem Movie gibt \\
	\hline
	Performer von Movie entfernen & In der Detailansicht eines Movies unter Performer auf das rote Kreutz neben dem Performer klicken, den man entfernen möchte & So kann man einen Performer wieder entfernen \\
	\hline
	Collection von Movie entfernen & In der Detailansicht eines Movies unter Collection auf das rote Kreutz neben der Collection klicken, die man entfernen möchte. & So kann man eine Collections wieder entfernen \\
	\hline
	Online Similar von Movie entfernen & In der Detailansicht eines Movies unter Online Similar auf das rote Kreutz neben dem Movie klicken, das man entfernen möchte. & So kann man ein Online Similar wieder entfernen \\
	\hline
\end{longtable}
%%\end{tabular}

\end{document}
%}}}







