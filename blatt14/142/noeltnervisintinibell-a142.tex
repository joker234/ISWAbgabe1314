\documentclass{beamer} %% normal document
%\documentclass[notes]{beamer} %% notes in normal document
%\documentclass[draft,notes]{beamer} %% draft with notes
%\documentclass[handout]{beamer} %% handout

\usepackage{fontspec, unicode-math, caption}
\usepackage[ngerman]{babel}
\usepackage{graphicx}
\usepackage{float}

%% usefull for handout with blank lines
%\usepackage{handoutWithNotes}
%\pgfpagesuselayout{2 on 1 with notes}[a4paper,border shrink=5mm]

%% usefull for presentation
%\setbeameroption{show notes on secondscreen=left}

\definecolor{bettergreen}{rgb}{.1,.7,.1}

\usetheme{Dresden}
\usecolortheme[named=bettergreen]{structure}
\useoutertheme{split}
\setbeamertemplate{caption}[numbered]
\captionsetup{labelformat=simple,font=scriptsize,labelfont=scriptsize}

% puts Frame numbers in Dresden template
\newcommand*\oldmacro{}%
\let\oldmacro\insertshorttitle%
\renewcommand*\insertshorttitle{%
	\oldmacro\hfill%
	\insertframenumber\,/\,\inserttotalframenumber}

\title[]{}

\author{
	Johannes Visintini, Philip Bell,\\
	Moritz Nöltner
}

\institute[IFI]{
	Vorlesung: Einführung in Software Engineering\\
	Institut für Informatik\\
	Universität Heidelberg
}
 
\begin{document}

	\begin{frame}
		\titlepage
		\note{ }
	\end{frame}

	\section{GUI}
	\subsection{Entwurf}
	\begin{frame}{GUI - Entwurf}
		\begin{figure}[H]
			\centering
			\includegraphics[width=\linewidth]{gui-mockup.png}
		\end{figure}
	\end{frame}

	\subsection{Simple Dialog}
	\begin{frame}{GUI - Simple Dialog}
		\begin{figure}[H]
			\centering
			\includegraphics[width=\linewidth]{simple-dialog.png}
		\end{figure}
	\end{frame}

	\subsection{Show Loaned Movies Dialog}
	\begin{frame}{GUI - Show Loaned Movies Dialog}
		\begin{figure}[H]
			\centering
			\includegraphics[width=\linewidth]{show-loaned-movies-dialog.png}
		\end{figure}
	\end{frame}

	\subsection{Add Performer Dialog}
	\begin{frame}{GUI - Add Performer Dialog}
		\begin{figure}[H]
			\centering
			\includegraphics[width=\linewidth]{add-performer-dialog.png}
		\end{figure}
	\end{frame}

	\subsection{Movie Wizard}
	\begin{frame}{GUI - Movie Wizard}
		\begin{figure}[H]
			\centering
			\includegraphics[width=\linewidth]{movie-wizard.png}
		\end{figure}
	\end{frame}

	\section{}
	\begin{frame}
		\center{\Huge ENDE}
	\end{frame}

\end{document}
