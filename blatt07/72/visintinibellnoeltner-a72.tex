%{{{ Heading Stuff
\documentclass[ngerman]{scrartcl}
\usepackage[left=1cm, right=1cm, top=1cm, bottom=1cm]{geometry}
\usepackage[ngerman]{babel}
\usepackage{fontspec}


\usepackage{multirow}
\usepackage{array}
\usepackage{pdfpages}

%%\documentclass[ngerman]{scrartcl}
%%\usepackage[ngerman]{babel}
%%\usepackage{fontspec}

%}}}


%{{{
\begin{document}

\author{Johannes Visintini, Philip Bell, Moritz Nöltner}
\title{Einführung in Software Engineering}
\subtitle{WS 13/14, Aufgabe 7.2}
\maketitle

\textheight = 650pt

\subsection*{Aufrufe von Oberflächenklassen an Operationen von Modellschichtklassen}
Es wird model.addMovie() aus der UI bzw. dem ActionListener aufgerufen.

\subsection*{Was bewirken diese?}

Diese Aufrufe sorgen dafür, dass in den Modelschichtklassen verschiedene Aufrufe zur Erstellung von Movies o.ä. aufgerufen werden.\\

Mehr haben wir nicht hingeschrieben, da die Aufgabe sich (unserer Auffassung nach) nur auf die Teilaufgaben 1. und 2. bzw. auf addMovie() bezieht.

\end{document}
%}}}







