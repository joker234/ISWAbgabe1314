\documentclass{scrartcl}

\usepackage[ngerman]{babel}
\usepackage{fontenc}
\usepackage[utf8]{inputenc}
%\usepackage{a4wide}
\usepackage[top=2cm, bottom=2cm, left=1cm, right=1cm]{geometry}


\begin{document}
\title{ISW - Aufgabe 1.3}
\author{Johannes Visintini}
\maketitle

\section*{Aufgabe 1.3}
\begin{tabular}{|p{3cm}|p{7cm}|p{7cm}|}\hline
	\textbf{Softwarefehler}
		& \textbf{Erklärung}
		& \textbf{Statisch/Dynamisch + Begründung}\\\hline
	Null-Pointer-Derefernzierung
		& Wenn ein Pointer auf keinen gültigen Speicherbereich zeigt und auf
			diesen Zugegriffen wird, wird eine “Null Exception” zurückgegeben.
		& Dynamisch — Wenn ein Pointer „Deklariert“ aber \textbf{nicht}
			„Initialisiert“ wird, ist ihm kein gültiger Speicherbereich
			zugewiesen. So kann es zu einer “Null Exception” kommen.\\\hline
	Vergleich oder Zuweisung
		& Wenn eigentlich etwas Zugewiesen werden sollte, dann aber ein
			Vergleich verwendet wurde („==“ statt „=“)
		& Statisch — Der Compiler kann diesen Fehler erkennen, da er aber nicht
			immer ein Fehler ist (sondern auch gewollt sein kann) gibt der
			Compiler eine Warnung aus.\\\hline
	fehlerhafter Arrayzugriff
		& Wenn auf das Array außerhalb des Arraybereichs zugegriffen werden
			soll. (z.B. auf „foo[12]“, wenn foo vorher als „int[10] foo“
			definiert wurde)
		& Dynamisch — weil der Arrayzugriffspunkt von Benutzereingaben abhängen
			kann\\\hline
	fehlendes volatil
		& Wenn ein Wert jedes Mal aus der Quelle gelesen werden muss (weil
			diese Quelle z.B. Live-Werte liefert, aus Sensoren o.ä.) und diese
			nicht gecached werden dürfen, muss ein volatil gesetzt sein
		& Statisch — Könnte ein Compiler erkennen. Dies tun aber nicht alle
			gängigen Compiler\\\hline
\end{tabular}
\end{document}
