%{{{ Heading Stuff
\documentclass[ngerman]{scrartcl}
\usepackage[left=1cm, right=1cm, top=1cm, bottom=1cm]{geometry}
\usepackage[ngerman]{babel}
\usepackage{fontspec}


\usepackage{multirow}
\usepackage{array}
\usepackage{pdfpages}

%%\documentclass[ngerman]{scrartcl}
%%\usepackage[ngerman]{babel}
%%\usepackage{fontspec}

%}}}


%{{{
\begin{document}

\author{Johannes Visintini, Philip Bell, Moritz Nöltner}
\title{Einführung in Software Engineering}
\subtitle{WS 13/14, Aufgabe 6.4}
\maketitle

\textheight = 650pt

\textbf{Was sind die wichtigsten Elemente der Sequenzdiagramme?}
\begin{itemize}
	\item[-] Beteiligte (Objekt) mit Lebenslinie
	\item[-] Pfeile (asynchrone-, synchrone-, Antwort-Nachricht)
	\item[-] Alternativen (alt)
	\item[-] Wiederholungen (Loop)
	\item[-] Objekterzeugung und -löschung
\end{itemize}

\textbf{Wozu werden Sequenzdiagramme vor allem verwendet?}\\
Sequenzdiagramme zeigen das Verhalten von Funktionen und Programmen. Sie
stellen dar wie die Objekte/Beteiligte miteinander interagieren. Es hilft
Systemfunktionen mit Hilfe von Objekten zu realisieren. Außerdem wird das
Zusammenspiel von mehreren Objekten, die durch Zustandsdiagramme dargestellt
sind, dargestellt.\\

\textbf{Was ist der Unterschied zwischen einem Sequenzdiagramm und einem Kommunikationsdiagramm?}
\begin{itemize}
	\item Objekte werden 2dimensional und nicht nur nebeneinander (mit Lebenslinie) dargestellt.
	\item Lebenslinien werden durch Nummerierungen ersetzt.
	\item einige Details (wie z.B. Löschungen) werden nicht explizit dargestellt.
\end{itemize}

\end{document}
%}}}







